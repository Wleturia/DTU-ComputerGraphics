\chapter{Week 1}

\section{Part 1}
The purpose of the lines that have been commented out in the appendix 1 is to
setup the projection of the camera. This defines a mapping between world space
and view space. By default - when these lines are commented out - this mapping
is defined by an identity matrix, which means that the viewer sees object located
within the $[0,1][0,1]$ intervals with respect to x an y coordinates.

gluOrtho2d as used below defines a scale and a translation so that the viewer sees object
located within $[-10,10][-10,10]$    


glMatrixMode is a primitive that selects the current matrix, so that OpenGL matrix
operations carried after are operated on the one selected (projection or modelview)

\begin{verbatim}
//glMatrixMode (GL_PROJECTION);
//glLoadIdentity ();
//gluOrtho2D (-10., 10., -10., 10.);
//glMatrixMode (GL_MODELVIEW);
\end{verbatim}

\section{Part 2}

Here are the lines modified as requested in the assignment
\begin{lstlisting}[caption=Snapshot from Part2.cpp]
glLoadIdentity ();
glTranslated(1.5,0,0);
glRotated(45, 0, 0, 1);
glTranslated(-1.5,0,0);
glColor3f(1.0,1.0,0.0);
glBegin (GL_POLYGON);
    glVertex2fv (V[0]);
    glVertex2fv (V[1]);
    glVertex2fv (V[2]);
    glVertex2fv (V[3]);
glEnd ();

glLoadIdentity();
glTranslated(6,7,0);
glBegin(GL_TRIANGLES);
    glColor3f (1.0, 0.0, 0.0);
    glVertex2f(2.0, 2.0);
    glColor3f (0.0, 1.0, 0.0);
    glVertex2f(5.0, 2.0);
    glColor3f (0.0, 0.0, 1.0);
    glVertex2f(3.5,5);
glEnd();
\end{lstlisting}

This gives the following result:

\image{Week01/Part2.png}{Output image of Part 2.}{0.5}{img:p2}


%\section{Part 3}



%\section{Part 4}


\section{Part 5}

\image{Week01/Part05.png}{Output image of Part 5.}{0.5}{img:p5}

\section{Part 6}


Given a viewport defined by the coordinates $[x_{w1},x_{w2},y_{w1},y_{w2}]$ and $[x_{v1},x_{v2},y_{v1},y_{v2}]$,
the matrix mapping a point $(x_{w},y_{w})$ to the point $(x_{v},y_{v})$
 is:


$$
\begin{pmatrix}
    A& 0& 0& tx \\
    0& B& 0& ty \\
    0& 0& 0& 0 \\
    0& 0& 0& 1 \\
\end{pmatrix}
$$

With the scaling components\\
\begin{align*}
    A &= \frac{1}{sx} = \frac{1}{(x_{min}-x_{u2})/(x_{v1}-x_{v2})} \\
    B &= \frac{1}{sy} = \frac{1}{(y_{min}-y_{u2})/(y_{v1}-y_{v2})}
\end{align*}
And the translation components\\
\begin{align*}
   t_x &= x_{v1}-x_{u2} \\
   t_y &= y_{v1}-y_{u2} 
\end{align*}

\section{Part 7}
The required transformations have been implemented in the function display below:
\begin{lstlisting}[caption=Snapshot from Part7.cpp]
void display (void) {
    glClear (GL_COLOR_BUFFER_BIT | GL_DEPTH_BUFFER_BIT);
    glColor3f (1.,1.,1.);
    // tansformation
    glTranslated(0,3,0);
    glRotated(30,0,1,0);
    glScaled(2,2,2);
    // draw the cube
    glutWireCube (1.);
    // draw the axis 
    glLoadIdentity();
    axis();
    glFlush ();
}
\end{lstlisting}

A translation matrix can is expressed as follows:
$$
\begin{pmatrix}
	1&	0& 	0&	tx\\
	0&	1&	0&	ty\\
	0&	0&	1&	tz\\
	0&	0&	0&	1
\end{pmatrix}
$$
A rotation around y is:
$$
\begin{pmatrix}
	cos(\theta)&	0& 	sin(\theta)&	0\\
	0&	        1&	0&	        0\\
	-sin(\theta)&	0&	cos(\theta)&	0\\
	0&	        0&	0&	        1\\
\end{pmatrix}
$$
And a scale is written as:
$$
\begin{pmatrix}
	sx	0 	0	0
	0	sy	0	0
	0	0	sz	0
	0	0	0	1
\end{pmatrix}
$$
witz sx, sy and sz the scaling factors along x, y and z.\\
~\\
Therefor, the matrix form of the transformations we used are expressed as follows:

glTranslated(0,3,0);
$$
\begin{pmatrix}
	1&	0& 	0&	0\\
	0&	1&	0&	3\\
	0&	0&	1&	0\\
	0&	0&	0&	1
\end{pmatrix}$$
glRotated(30,0,1,0);
$$\begin{pmatrix}
	cos(30)&	0& 	sin(30)&	0\\
	0&	        1&	0&	        0\\
	-sin(30)&	0&	cos(30)&	0\\
	0&	        0&	0&	        1\\
\end{pmatrix}$$

glScaled(2,2,2);
$$\begin{pmatrix}
	2&	0& 	0&	0\\
	0&	2&	0&	0\\
	0&	0&	2&	0\\
	0&	0&	0&	1\\
\end{pmatrix}$$

the final modelview matrix is the multiplication of the three previous matrices.

\image{Week01/Part7.png}{Output image of Part 7.}{0.5}{img:p5}

\section{Part 8}

\image{Week01/Part8a.png}{Front perspective view.}{0.5}{img:p5}
\image{Week01/Part8.png}{X perspective view.}{0.5}{img:p5}

