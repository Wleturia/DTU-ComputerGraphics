\chapter{Week 5 - Texture mapping}

\section{Part 1}
  The difference between activating or not the \emph{Depth Test} is shown in Images \ref{img:01-01}
  and \ref{img:01-02}. The difference is that with Depth test off all the meshes are drawn in order,
  ignoring the fact that the objects can hide others if placed nearer the viewer, while if that is on 
  then an object hidden by another won't be drawn.

  In the program the draw order is: Axis, Green Plane, Light Blue Cube.

  \image{Week05/Ex_01_-_No_Clip}
        {First Program with Depth Test Deactivated}
        {0.8}
        {img:01-01}

  \image{Week05/Ex_01_-_Clip}
        {First Program with Depth Test Activated}
        {0.8}
        {img:01-02}


\section{Part 2}

Here are the lines modified as requested in the assignment
\begin{lstlisting}[caption=Snapshot from Part2.cpp]
glLoadIdentity ();
glTranslated(1.5,0,0);
glRotated(45, 0, 0, 1);
glTranslated(-1.5,0,0);
glColor3f(1.0,1.0,0.0);
glBegin (GL_POLYGON);
    glVertex2fv (V[0]);
    glVertex2fv (V[1]);
    glVertex2fv (V[2]);
    glVertex2fv (V[3]);
glEnd ();

glLoadIdentity();
glTranslated(6,7,0);
glBegin(GL_TRIANGLES);
    glColor3f (1.0, 0.0, 0.0);
    glVertex2f(2.0, 2.0);
    glColor3f (0.0, 1.0, 0.0);
    glVertex2f(5.0, 2.0);
    glColor3f (0.0, 0.0, 1.0);
    glVertex2f(3.5,5);
glEnd();
\end{lstlisting}

This gives the following result:

\image{Week05/Ex_02_-_Clipping.png}{Output image of Part 2.}{0.5}{img:p52}


\section{Part 3}

Playing with the filtering techniques we obtain the results in figures
\ref{img:p531},\ref{img:p532},\ref{img:p533},\ref{img:p534},\ref{img:p535} and \ref{img:p536}. 

\image{Week05/Ex_03_-_Linear.png}{Linear sampling without mipmaps.}{0.5}{img:p531}
\image{Week05/Ex_03_-_Linear_with_Linear_Mipmaps.png}{Linear sampling with mipmaps.}{0.5}{img:p532}
\image{Week05/Ex_03_-_Linear_with_Nearest_Mipmaps.png}{Linear sampling with nearest mipmaps.}{0.5}{img:p533}
\image{Week05/Ex_03_-_Nearest.png}{Nearest sampling without mipmaps.}{0.5}{img:p534}
\image{Week05/Ex_03_-_Nearest_with_Linear_Mipmaps.png}{Nearest sampling with linear mipmaps.}{0.5}{img:p535}
\image{Week05/Ex_03_-_Nearest_with_Nearest_Mipmaps.png}{Nearest sampling with nearest mipmaps.}{0.5}{img:p536}

Nearest sampling simply takes the colour of the learest texel in the texure, without accounting
for the surrounding texel. This method is fast but produces some alisaing, especially
when the triangle is not facing the viewer. This sampling method can show some noisy
artifacts when the size of the texels are smaller than the one of the pixels on the
screen, because some of them are just missed.
Linear sampling, however, uses linear interpolation between the closest four closest
texels, which produces smoother results. When upscaling and downscaling the result is
blured a bit.

Mipmaping is used to precompute downscaled versions of the texture in order to
optimize speed and reduce aliasing.
Speed optimization comes from the fact that linear sampling has less fragments to
sample when the triangle is far away (down scaling). Aliasing is reduced because
the precomputation can use better scaling methods since it does not have to be
computed all the time. 

\section{Part 4}

\image{Week05/Ex_04_-_Direct_Map.png}{ Direct mapping. No fragment is sampled outside the texture's range.}{0.5}{img:p541}
\image{Week05/Ex_04_-_Bidirectional_Clamp.png}{Bidirectional clamping. The texture borders of the texture are stretched when sampling out of the texture range.}{0.5}{img:p542}
\image{Week05/Ex_04_-_S_Repeat.png}{Repeated texture along S.}{0.5}{img:p543}
\image{Week05/Ex_04_-_T_Repeat.png}{Repeated texture along T.}{0.5}{img:p544}
\image{Week05/Ex_04_-_Repeated_Texture_1-10.png}{Repeated texture (in both dimensions).}{0.5}{img:p545}

\section{Part 5}

\image{Week05/Ex_05_-_Rotated_Texture.png}{Texture rotated by 45°.}{0.5}{img:p551}
\image{Week05/Ex_05_-_Rotated_and_scaled_down_on_X_axis.png}{Texture rotated 45° and scaled in one direction.}{0.5}{img:p552}
\image{Week05/Ex_05_-_Rotated_and_scaled_175_on_both_axis.png}{Texture rotated 45° and scaled in both directions.}{0.5}{img:p553}


\section{Part 6}

\image{Week05/Ex_06_-_SRS_Texture_Mapping.png}{Using an image from disk as texture}{0.5}{img:p561}
