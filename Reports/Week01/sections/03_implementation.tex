The last part of this project was its implementation in \nxtOSEK and
testing its behaviour in the real world.

\section{Transforming the controller from the Discrete version to the
  Algorithmic one}

  To implement the controller in \nxtOSEK{} the same methos used in the
  previou report was used, the steps are:

  \begin{align*}
    \Omega(z) &= E(z) \cdot{} \frac{z \cdot (2 + Z_1 \cdot T_O) - 2 + Z_1
    \cdot{} T_O }{z \cdot{} (2 + P_1 \cdot T_O) - 2 + P_1
    \cdot{} T_O } \\
    \Omega(z) \cdot (z \cdot{} (2 + P_1 \cdot T_O) - 2 + P_1
    \cdot{} T_O) &= E(z) \cdot{} z \cdot (2 + Z_1 \cdot T_O) - 2 + Z_1
    \cdot{} T_O \\
    \omega(k+1) \cdot (2 + P_1 \cdot T_O) &= e(k + 1) \cdot (2 + Z_1 \cdot
    T_O) + e(k) \cdot (T_O \cdot Z_1 - 2) - \omega(k) \cdot (T_O \cdot Z_1
    - 2) \\
    \omega(k) &= \frac{e(k) \cdot (2 + Z_1 \cdot
    T_O) + e(k - 1) \cdot (T_O \cdot Z_1 - 2) - \omega(k - 1) \cdot (T_O \cdot Z_1
    - 2)}{2 + P_1 \cdot T_O}
  \end{align*}
  
  The last step of this equation was the one used to implement the controller on the \nxt{}

\section{Implementation of the Controller}

  The controller was implemented in \nxtOSEK{} using the application
  already developed for the previous part of the project and adding one
  task, \techname{ADA\_Eyes}.

  This task computes the distance and the next angular speed of the vehicle
  every $30 ms$ using the algorithmic version of the controller and the
  formula shown in section \ref{eqn:MathDist}. 

\section{Real world results}

  The controller was tested with various starting positions but it behaved
  correctly only in the range between $0.13 m$ and $0.7 m$, since it was
  developed to be working by starting from a maximum distance of $0.3 m$
  from the target (which was $0.4 m$).

  Figures \ref{img:LiberaMeNear} and \ref{img:LiberaMeFar} show how the
  system behaved when starting, respectively, almost against the wall and
  $0.7 m$ away from it.

  \begin{figure}[h!]
    \centering
    \includegraphics[width=\textwidth]{Images/ADA_Near_Real_vs_Simulated.pdf}
    \caption{Performance of the Controller when placed almost against the
      wall compared to the simulated behaviour\label{img:LiberaMeNear}}
  \end{figure}

  \begin{figure}[h!]
    \centering
    \includegraphics[width=\textwidth]{Images/ADA_Far_Real_vs_Simulated.pdf}
    \caption{Performance of the Controller when placed at $0.7 m$ away from
      the wall compared to the simulated behaviour\label{img:LiberaMeFar}}
  \end{figure}

\section{Issues with real world implementation}

  Along with the working range of the sensors there were also other issues
  with the implementation of the controller, which hindered its
  performances:

  \begin{itemize}
    \item{} If the batteries were not charged at least at 75\% of their
      maximum charge then the motors didn't have enough power to move the
      entire vehicle. This lead to a batch of tests with wrong results,
      since the system was working correctly up to the day before and the
      code was not changed
    \item{} The wheels didn't have a grip good enough for the floor on
      which the vehicle was tested. This lead to tests with wrong results,
      since the wheels skitted on the floor. Since the skitting was
      non-evenly the angular speed was often wrong. This was fixed by
      lowering the gain of the motors and by cleaning the wheels from the
      dust before each test.
    \item{} Apparently the device hook \codeconst{ecrobot_device_initialize()}
      is not called every time the system is restarted but instead is
      called only when \nxtOSEK{} is loaded into RAM. This lead to wrong
      initial computations for the controller, requiring high angular
      speeds right from the begin of the execution. To avoid this behaviour
      the \nxt{} is shut down each time to allow a correct initialisation
      of the variables and structures.
  \end{itemize}
