\chapter{Week 5}

\section{Part 1}
  The difference between activating or not the \emph{Depth Test} is shown in Images \ref{img:01-01}
  and \ref{img:01-02}. The difference is that with Depth test off all the meshes are drawn in order,
  ignoring the fact that the objects can hide others if placed nearer the viewer, while if that is on 
  then an object hidden by another won't be drawn.

  In the program the draw order is: Axis, Green Plane, Light Blue Cube.

  \image{Week05/Ex_01_-_No_Clip}
        {First Program with Depth Test Deactivated}
        {0.8}
        {img:01-01}

  \image{Week05/Ex_01_-_Clip}
        {First Program with Depth Test Activated}
        {0.8}
        {img:01-02}


\section{Part 2}

Here are the lines modified as requested in the assignment
\begin{lstlisting}[caption=Snapshot from Part2.cpp]
glLoadIdentity ();
glTranslated(1.5,0,0);
glRotated(45, 0, 0, 1);
glTranslated(-1.5,0,0);
glColor3f(1.0,1.0,0.0);
glBegin (GL_POLYGON);
    glVertex2fv (V[0]);
    glVertex2fv (V[1]);
    glVertex2fv (V[2]);
    glVertex2fv (V[3]);
glEnd ();

glLoadIdentity();
glTranslated(6,7,0);
glBegin(GL_TRIANGLES);
    glColor3f (1.0, 0.0, 0.0);
    glVertex2f(2.0, 2.0);
    glColor3f (0.0, 1.0, 0.0);
    glVertex2f(5.0, 2.0);
    glColor3f (0.0, 0.0, 1.0);
    glVertex2f(3.5,5);
glEnd();
\end{lstlisting}

This gives the following result:

\image{Week01/Part2.png}{Output image of Part 2.}{0.5}{img:p2}


%\section{Part 3}



%\section{Part 4}


\section{Part 5}

\image{Week01/Part05.png}{Output image of Part 5.}{0.5}{img:p5}

\section{Part 6}


Given a viewport defined by the coordinates $[x_{w1},x_{w2},y_{w1},y_{w2}]$ and $[x_{v1},x_{v2},y_{v1},y_{v2}]$,
the matrix mapping a point $(x_{w},y_{w})$ to the point $(x_{v},y_{v})$
 is:


$$
\begin{pmatrix}
    A& 0& 0& tx \\
    0& B& 0& ty \\
    0& 0& 0& 0 \\
    0& 0& 0& 1 \\
\end{pmatrix}
$$

With the scaling components\\
\begin{align*}
    A &= \frac{1}{sx} = \frac{1}{(x_{min}-x_{u2})/(x_{v1}-x_{v2})} \\
    B &= \frac{1}{sy} = \frac{1}{(y_{min}-y_{u2})/(y_{v1}-y_{v2})}
\end{align*}
And the translation components\\
\begin{align*}
   t_x &= x_{v1}-x_{u2} \\
   t_y &= y_{v1}-y_{u2} 
\end{align*}

\section{Part 7}
The required transformations have been implemented in the function display below:
\begin{lstlisting}[caption=Snapshot from Part7.cpp]
void display (void) {
    glClear (GL_COLOR_BUFFER_BIT | GL_DEPTH_BUFFER_BIT);
    glColor3f (1.,1.,1.);
    // tansformation
    glTranslated(0,3,0);
    glRotated(30,0,1,0);
    glScaled(2,2,2);
    // draw the cube
    glutWireCube (1.);
    // draw the axis 
    glLoadIdentity();
    axis();
    glFlush ();
}
\end{lstlisting}

A translation matrix can is expressed as follows:
$$
\begin{pmatrix}
	1&	0& 	0&	tx\\
	0&	1&	0&	ty\\
	0&	0&	1&	tz\\
	0&	0&	0&	1
\end{pmatrix}
$$
A rotation around y is:
$$
\begin{pmatrix}
	cos(\theta)&	0& 	sin(\theta)&	0\\
	0&	        1&	0&	        0\\
	-sin(\theta)&	0&	cos(\theta)&	0\\
	0&	        0&	0&	        1\\
\end{pmatrix}
$$
And a scale is written as:
$$
\begin{pmatrix}
	sx	0 	0	0
	0	sy	0	0
	0	0	sz	0
	0	0	0	1
\end{pmatrix}
$$
witz sx, sy and sz the scaling factors along x, y and z.\\
~\\
Therefor, the matrix form of the transformations we used are expressed as follows:

glTranslated(0,3,0);
$$
\begin{pmatrix}
	1&	0& 	0&	0\\
	0&	1&	0&	3\\
	0&	0&	1&	0\\
	0&	0&	0&	1
\end{pmatrix}$$
glRotated(30,0,1,0);
$$\begin{pmatrix}
	cos(30)&	0& 	sin(30)&	0\\
	0&	        1&	0&	        0\\
	-sin(30)&	0&	cos(30)&	0\\
	0&	        0&	0&	        1\\
\end{pmatrix}$$

glScaled(2,2,2);
$$\begin{pmatrix}
	2&	0& 	0&	0\\
	0&	2&	0&	0\\
	0&	0&	2&	0\\
	0&	0&	0&	1\\
\end{pmatrix}$$

the final modelview matrix is the multiplication of the three previous matrices.


\section{Part 8}

% TODO images or something ??
