\chapter{Week 3}

\section{Part 1}

A rasterization process like OpenGL uses cannot easily simulate real light, and
more importantly indirect lighting. In real life when sun rays enter a room, they
bounce on walls, which provides lighting for parts that dont directly face the light
sources. To emulate the effect of indirect lighting, OpenGL includes the ambient
term in its lighting equation. What it does is basically apply a base colour to
every vertice regardless of the orientaion of its normal. 
~\\
~\\
TODO: 2nd ambiant component ??

%\image{Week03/Part1.png}{Output image of Part 1.}{0.5}{img:w2p1}

\section{Part 2}

\section{Part 3}
\begin{itemize}
    \item The distance between the eye and the object is not taken into account by the phong
    equation - only the direction is - therefore, puting the eye at infinity does not
    change the lighting of the object.
    \item The vew vector - vector between the eye and the object, thus dependent on the
    eye position - influences the specular component.
    The object will not get illuminated - except for ambiant - by the light source if the later is placed at
    infinity, because the equation includes an attenuation term accounting for the
    distance between the objetc and the light source.
    \item Binding the light source to the eye point results illumination only dependent on
    the viewer, with the most illuminated parts being the one with normals directed
    toward the viewer. For instance turning around a sphere would give the same
    result whichever direction the viewer is looking the sphere from.
    \item Directional lights correspond to point lights placed at infinity without attenuation
    term. This kind of light are generally used to simulate the effect of powerful
    and very distant light sources like the sun.
\end{itemize}
\section{Part 4}


\begin{itemize}
    \item Flat shading renders each polygon of an object a single color, using
    the one of the first vertex of the polygon.
    \item Gouraud shading offers a smoother result as it computes the lighting of
    every vertex and interpolates the values across the polygon.
    \item Phong shading samples the amount of light of the object at several points
    within the polygon, giving more precise results than gouraud shading.
\end{itemize}
There's of course a tradeof between realism and computation time, that must be taken
into account when shoosing between these techniques (at least in low end hardware
like mobile devices as desktop graphics use more and more complex per-pixel deffered
shading technics).\\
Among the three methods mentionned above, phong shading is the best to render highlights
because highlights are mostly small areas where the amount of light is intense,
therefore the sampling density counts. Gouraud shading can miss small highlight
positionned between two vertice, whereas Phong samples the quantity of light between
vertice. \\
Specular reflections depend on the position of the eye so yes, the shading
depends also on the position on the screen if the position on the screen influences
the direction of the vector coming from the eye to the object (I am not sure I understood the question though).

