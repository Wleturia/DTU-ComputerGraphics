\chapter{Week 6 - Vertex and fragment shaders}

We used a simple scene with a rotating cube available in campusnet (02561-06-00.2011.zip)
as it provides a simple mesh as well as shader initialization out of the box.

Gauraud shading computed in the vertex shader is rendered as seen in figure \ref{gouraud}.

Per vertex shading is not perfect on a cube mesh, because the latter has a very low
tesselation so small details such as the highlight are missed most of the time.

The result can be improved however, by computing the highlight in the fragment shader.
We implemented Phong shading by keeping the ambient and diffuse computation in the
vertex shader and computing only the specular term in the fragment shader. The result
can be seen in figure \ref{phong}.

We also added texturing to the shader as shown in figure \ref{textured}. This required
to add a texture sampler uniform and texture coordinates vertex attributes in the C++
code.   

\image{Week06/Gouraud.png}{Gauraud shading (per vertex lighting).}{0.5}{gouraud}
\image{Week06/Phong.png}{Phong shading (per fragment lighting).}{0.5}{phong}
\image{Week06/Textured.png}{Textured cube (with specular highlight).}{0.5}{textured}

The sources of the shader for Phong shading and Texture mapping are in figure \ref{vs} and \ref{fs}.
There is only one line to change in the fragment shader to switch between the two.
Normally it should have been done in two different files so as to avoid useless
computations and inputs, but it is easier to present this and avoids long code listings.

\fcode{Week06/Phong.vert}{java}{Vertex Shader for phong and Texture shading}{vs}
\fcode{Week06/Phong.frag}{java}{Fragment Shader for phong and Texture shading}{fs}
